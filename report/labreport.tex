\documentclass[10pt,a4paper]{labreport}
\usepackage{csquotes}
\usepackage{titlesec}
\usepackage{ragged2e}
\usepackage{siunitx}
\usepackage{setspace}
\usepackage{longtable}
\usepackage{rotating}
\usepackage{xurl}
\usepackage{physics}
\usepackage{caption}
\usepackage{wrapfig}
\usepackage{tabularray}
\usepackage{fancyhdr}
\usepackage{subcaption}
\usepackage{lscape}
\usepackage{tensor}
\usepackage{multirow}
\usepackage{chemformula}
\usepackage[gen]{eurosym}
\usepackage{float}
\usepackage{bm}
\usepackage{lipsum}
\usepackage{parskip}
\usepackage{booktabs}
\usepackage{enumerate}
\usepackage[justification=justified]{caption}
\usepackage[nottoc]{tocbibind}
\usepackage{hyperref}
%  \usepackage[
% backend=biber,
% style=chem-acs,articletitle=true,doi=true]{biblatex}
% %\addbibresource{references.bib}





\title{Nanoscale Material Modeling
\\
\normalsize{Week 3}} % Main title and sub title. 

\author{Ilija A. Gjerapić, S4437586; \href{mailto:i.a.gjerapic@student.rug.nl}{i.a.gjerapic@student.rug.nl}; \href{https://github.com/igjerapic/nmm-week3/}{@github} } % Name, student number, email

\supervisors{prof. dr. A. Giuntoli, prof. dr. J. Slawinska}

\begin{document}


\maketitle



  

\thispagestyle{firststyle}
\newpage
\section{Assignment 1: DFT calculation of WTe2 monolayer}
Self-consistant field calculation
\begin{itemize}
    \item Bravais lattice is rectangular: The two W atoms (grey balls) do not lie in the x-y plane, but have an angle of 4.27deg between them. Makes sense due to {\color{red} MATCHING WITH REFERENCES}
  
    \item Supercell constructed using the Cell dimensions ?Flag? with dimensions 3.5 x 6.28 x 30 \AA. 
    \item Distance between monolayers is 30 \AA, as expected from supercell
    \item Running for 3 minutes showed approx 40 seconds per iteration. The run time was then set to 30 minutes assuming that it would take max 15 iterations each of 1 min long. 
  \end{itemize}

  Band structure
  \begin{itemize}
    \item No nscf was necessary as the kpoints were interpolated with gaussian smearing
    \item Show image of points taken using xcrysden
    \item for plotting check out \url{https://github.com/quantumNerd/Quantum-Espresso-Tutorial-2019-Projects/tree/master/tools/plot_band} 
  \end{itemize}
  The visualization of the \texttt{Si.sci.in} input file is shown in Figure \ref{fig:ass1_cryst}. 
  \begin{figure}[h]
    \centering 
    \includegraphics[width = 0.7\textwidth]{example-image-a}
    \caption{\textbf{(a)} The conventional FCC unit cell with a two atom basis. The lattice constant was found to be 5.43 \AA. with a bond angle of 90$^\circ$. (b) The primitive unit cell obtained from the input file. A lattice constant of 3.8396 {\AA} was found, with a bond angle of 60$^\circ$.}
    \label{fig:ass1_cryst}
  \end{figure}

\newpage
\section{Assignment 2: Graphene}
\begin{itemize}
  \item graphene supercell using graphite structure from week2
  \item Use xcrysden to inspect it
  \item band structure along M-G-K-M direction. Compare with literature 
\end{itemize}


\section{Assignment 3: Gr/WTe2 Heterostructure}
3a)
\begin{itemize}
  \item Analysis of wte2-gr.relax.in file
  \item Analysis of respective structure:
  \begin{itemize}
    \item hexagonal or rectangular: 
    
    Was found to be rectangular 
    \item twisting angle
    \item number of graphene unit cells
  \end{itemize}
\end{itemize}

3b)
\begin{itemize}
  \item Use own files for graphene and WTe2 monolayers to create heterostructure yourself
  \item Open files using xcrysden and save as xsf. Then check out \url{https://www.youtube.com/watch?v=vAc_vlSfQT0} to generate heterostructure 
\end{itemize}



\newpage
% \printbibliography

% \begin{appendices}
%   \input{Appendix}
% \end{appendices}

\end{document}



